\section{Conclusion and Future Work}
We have classified Chisel hardware construction into three different
levels. The first level of Chisel hardware construction has users
working directly with {\tt Nodes} to construct hardware. At the second level
of hardware construction, Chisel {\tt Nodes} are abstracted away into
Scala functions. We present examples of two useful hardware facilities
{\tt ListLookup} and {\tt Vec} implemented at both levels to highlight
the advantages and disadvantages of each level. Finally, the third
level of hardware construction has users writing elaboration time
functions that the Chisel compiler invokes to build hardware. We
present Chisel {\tt when} statements, SRAM backends, and an automatic
pipeline synthesis tool as examples of hardware construction at this
level.

In future work, we would examine other ways of constructing hardware
in Chisel. One possible way to further raise the hardware construction
abstraction level would be to implement a process language DSL in Chisel
for writing state machines. We would also explore other possible
use cases of Chisel's elaboration interface such as generating
multithreaded datapaths and FAME~\cite{fame} transformations.
