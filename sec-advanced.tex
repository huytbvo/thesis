\section{Advanced Hardware Construction in Chisel}
\label{sec:advanced}
Another way to support hardware facilities in Chisel, as opposed to
building new nodes, is to write a Scala program that aggregates the
features of Chisel from Chapter~\ref{sec:chisel} into a directed graph
of nodes that implement the same hardware functionality. This approach
moves the complexity of the hardware facility into the Scala graph but
makes the hardware facility much more accessible.
Unlike the approach from Chapter~\ref{sec:basic}, the
hardware facilities are now built out of basic Chisel {\tt Nodes} so
there is no dependence on backend support and no designer effort is
needed to target different backends. Users can fall back to this style
of hardware construction if they are targeting a backend that does
not have support for the hardware facility they are using. The
downside is that the resulting Chisel graph is much denser which can
lengthen the time spent in elaboration. We are also giving up on any
optimization the backend synthesis tool could provide for supported
hardware facilities.

The hardware facilities built in this section will use the same syntax
as their counterparts from Chapter~\ref{sec:basic}. The difference now
is that we are reusing basic Chisel {\tt Nodes} and not implementing
new {\tt Nodes}. This chapter will show how to implement ListLookup
and Vec using only the basic Chisel nodes presented in
Chapter~\ref{sec:chisel}.

\subsection{ListLookup}
\begin{figure}[htb]
\centering
\includegraphics[width=0.7\textwidth]{figures/listlookupscala.pdf}
\caption{ListLookup as Scala code and basic Chisel nodes}
\label{fig:llscala}
\end{figure}

The syntax for ListLookup remains the same. Users provide an address
for lookup and a list of size $m$ of tuples {\tt ($a_i$, $lits_i$)}
where {\tt $a_i$} is the address to match the lookup address against and
{\tt $lits_i$} is a list of size $n$ of literals to return when there
is a match. The ListLookup will then construct the graph in
Figure~\ref{fig:llscala}. The function returns a list of {\tt Bits}
{\tt $b_1$, ... $b_n$} who will take on the values of {\tt $lits_i$} if
{\tt $a_i$} matches the lookup address.

ListLookup graphs are constructed as follows. Given the input list of
tuples {\tt ($a_i$, $lits_i$)}, create $n$ list of $m$ tuples
{\tt $( (a_1, l_{1, k}), ... (a_j, l_{j, k}), ..., (a_m, l_{m, k})
  )$}. Each of these list represents a mux chain and there are $n$
such list ($k$ ranges from 1 to $n$). The mux chains are constructed
using {\tt $a_i$ === addr} as the select signal to the corresponding
mux. If the address matches, then literal {\tt $l_{i, k}$} is outputted
for Bits $b_k$.

\subsection{Vec}
\begin{figure}[htb]
\centering
\includegraphics[width=0.5\textwidth]{figures/vecscala.pdf}
\caption{Vec as Scala code and basic Chisel nodes}
\label{fig:vecscala}
\end{figure}

Figure~\ref{fig:vecscala} shows the resulting graph from building a
Vec using only basic Chisel nodes. Note that Vec is now a Scala array
of {\tt Bits} and is not a {\tt Node}. The syntax for constructing a
Vec is unchanged; users instantiate the Vec with the desired
size. Since the Vec is now a Scala array, users populate the Vec using
Scala array syntax, e.g.
\begin{align*}
  \text{ \tt{v(1) = Bits(width=32)} }
\end{align*}
where {\tt v} is the name of the Vec. The syntax for reading and
writing to the Vec is also unchanged. The difference is that reads
and writes will now construct the graph in Figure~\ref{fig:vecscala}
instead of instantiating special read and write nodes.

\textbf{Reads} A Vec can be indexed with either Scala integer or a
Chisel {\tt Node}. If a Scala integer is used, no {\tt Nodes} are
created and the element at that index is returned. If a Chisel 
{\tt Node} is used, then a Mux chain is built to mux out the correct
element of the Vec. The select signal to the mux is {\tt addr === idx}
where {\tt addr} is the address and {\tt idx} ranges from zero to
length of the Vec minus 1. When {\tt addr} matches an {\tt idx}, the
element at {\tt v(idx)} is outputted.

\textbf{Writes} Like reads, a Vec can be written to using either a
Scala integer or a Chisel {\tt Node} for indexing. If a Scala integer
is used, then the corresponding element is accessed and written to and
no nodes are constructed. If a Chisel {\tt Node} is used, then a mux
is placed in front of each element in the Vec. The select signal to
the mux is {\tt waddr ==== idx} where {\tt waddr} is the input write
address and {\tt idx} ranges from zero to the length of the Vec minus
1. If the write address matches the {\tt idx}, then the element at
{\tt v(idx)} is written to with the write data, otherwise {\tt v(idx)}
is written to with the default data. If the {\tt Vec} contains 
{\tt Bits} then the user supplies the default data. If the {\tt Vec}
contains {\tt Reg} nodes then no default signal is needed.
